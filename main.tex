\documentclass[12pt]{article}

% PDF Layout
\usepackage[a4paper,left=2cm,right=2cm,bottom=3cm,top=2.5cm]{geometry}

% Standard Packages
\usepackage[utf8]{inputenc}
\usepackage[T1]{fontenc}

\usepackage{hyperref}
\hypersetup{
	pdftitle={Note Template},
	pdfauthor={},
	colorlinks=true,
	linkcolor=black,
	citecolor=blue,
	pdfstartview=FitH,
	pdfpagemode=UseOutlines,
	pdfpagelayout=OneColumn,
}

% Personal Macros
\usepackage{em-mathtools}

\renewcommand{\qedsymbol}{$\blacksquare$}

% Bibliography Package
\usepackage[style=alphabetic,maxbibnames=6]{biblatex}
\addbibresource{formalising-wp-groups.bib}

\usepackage[capitalise]{cleveref}

\title{
  Project Title Here
}
\author{
  Author Name \large{email@email.com}
}

\begin{document}

\maketitle

\textcolor{red}{Probably want to keep it about 2 and a half pages (excluding references) unless there is a minimum word requirement. Page 1 should focus on the problem, page 2 should focus on methods and timeline.}

\section{Introduction}\label{sec:intro}
[maximum 200 words] \\

\noindent \textcolor{red}{Project description including: Aims, significance, and expected outcome of the project.}

\begin{enumerate}
  \item Brief problem description and project introduction.
  \item Aims and objectives of the research project (keep vague).
  \item Research significance and applications, you can talk about PQC, language verification, etc.
  \item Expected results... we expect to reimplement important theorems surrounding the word problem for groups...
\end{enumerate}

\section{Background}\label{sec:background}
[maximum 200 words] \\

\begin{itemize}
    \item Summary of relevant literature and theoretical framework
    \item Identification of gaps or areas for further research
    \item Explanation of how the proposed research fills those gaps or contributes to the existing knowledge
\end{itemize}

\paragraph{Preliminaries.} A group $G$ is said to be finitely generated if it's generating set is finite.

\paragraph{The Word Problem for Groups.} Given a group $G$ and a generating set $X$ for $G$, the word problem for $G$ and $X$ denoted $\WP(G,X)$, asks to decide whether,

\textcolor{blue}{Sinclair}

\begin{theorem}[{\cite[Theorem~3.4.1]{HoltEtAl2017}}]\label{thm:finite-groups-dfa-wp}
  Given a finitely generated group $G$ with generating set $X$, $\WP(G,X)$ is regular if and only if $G$ is finite.
\end{theorem}

\paragraph{Formal Verification in Lean} Many recent projects have investigated formalising mathematical theories in lean.
\textcolor{blue}{Aiden}

\section{Feasibility Study}
[maximum 200 words]
\newline

\textcolor{Feasibility study in terms of computing \& time resources, team’s capability, and skill sets.}

\textcolor{blue}{Aiden/Anyone}

\begin{itemize}
    \item Data collection methods (e.g., surveys, interviews, experiments, publicly available datasets)
    \item Data analysis procedures
    \item Solution design \& implementation
    \item Experimental verification
    \item Ethical considerations (e.g., informed consent, confidentiality)
\end{itemize}


\section{Project Management Plan}
%%%%% Anyone/Vikram %%%%%


\begin{itemize}
    \item Timeline \& Milestones.
    \item Roles \& Responsibilities.
    \item Budget \& Resources.
    \item Risk Management.
    \item An efficient communication plan.
\end{itemize}

\printbibliography

\textcolor{red}{List of all sources cited in the proposal, following a specific citation style (e.g., APA, MLA).}

% In~\cref{subsec:aims}. In~\cref{lem:finite-groups-dfa-wp}.

% Citation goes here\autocite{Author2001}
% Citation goes here\cite{Author2001}
% Author had the result P=NP\cite{PaperHere}

\end{document}
